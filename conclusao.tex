\chapter{Conclusão}
\label{cap:conclusao}

A mortalidade infantil é um problema bastante sério que afeta diretamente na saúde da população.
Ela pode ser verificada mais acentuadamente em regiões mais pobres, devido à falta de assistência, deficiência
hospitalar, contaminação de alimentos e água, etc.

O resultado deste trabalho corrobora o de alguns anteriores, como o de \cite{teixeira2011analise}, que
discorre sobre o impacto da cobertura de esgotamento sanitário na taxa de mortalidade infantil. Discorremos
sobre a relação inversa entre a porcentagem de cobertura de esgotamento sanitário e a taxa de mortalidade
infantil.

Existem evidências para que possamos aceitar o modelo como estatisticamente adequado nos quesitos de
significância dos coeficientes das variáveis independentes da regressão (As estatísticas t e F para o seu
conjunto possuem indícios de validade a 95\% de confiança), homocedasticidade e normalidade.

Este trabalho serve de incentivo para estudos maiores e robustos sobre o tema, utilizando amostras
mais representativas da população brasileira e técnicas estatísticas eficientes para oferecer respostas
completas sobre a questão da mortalidade infantil, suas causas e consequências; e no futuro contribuir
para ações governamentais com impactos mais contundentes no problema.
