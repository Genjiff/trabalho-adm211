\chapter{Desenvolvimento da regressão múltipla}
Foi feita uma análise de regressão linear múltipla sobre os valores da taxa de mortalidade infantil (menores de 1 ano) por 1000 nascidos - variável dependente -, da proporção
da população servida por esgotamento sanitário e pela proporção da população servida por coleta de lixo - variáveis independentes - a partir dos dados de todos os estados brasileiros
e do Distrito Federal. Os dados foram retirados do site do IDB \cite{idb} e são referentes ao ano de 2011. A regressão é apresentada abaixo, com $\alpha$ = 0.5:
\newline
\begin{table}[h]
\centering
\resizebox{\textwidth}{!}{%
\begin{tabular}{|l|l|l|l|l|l|}
\hline
Estados             & \multicolumn{1}{c|}{Taxa de mortalidade infantil} & \multicolumn{1}{c|}{População coberta por esgotamento} & \multicolumn{1}{c|}{População servida por coleta de lixo} & \multicolumn{1}{c|}{Taxa de mortalidade calculada} & \multicolumn{1}{c|}{Erro} \\ \hline
Acre                & 18.5                                              & 43.46                                                 & 78,21                                                     & 18,898548333                                       & -0,398548333              \\ \hline
Alagoas             & 17                                                & 45.7                                                  & 76,11                                                     & 19,0433062994                                      & -2,0433062994             \\ \hline
Amapá               & 24.1                                              & 56.55                                                 & 87,87                                                     & 16,4955285743                                      & 7,6044714257              \\ \hline
Amazonas            & 20                                                & 61.55                                                 & 78,59                                                     & 17,4989738524                                      & 2,5010261476              \\ \hline
Bahia               & 20.1                                              & 57.52                                                 & 77,83                                                     & 17,9108085623                                      & 2,1891914377              \\ \hline
Ceará               & 15.2                                              & 52.53                                                 & 74,19                                                     & 18,82056665                                        & -3,62056665               \\ \hline
Distrito Federal    & 12.1                                              & 96                                                    & 97,98                                                     & 12,068471693                                       & 0,031528307               \\ \hline
Espírito Santo      & 11.7                                              & 82.79                                                 & 90,03                                                     & 14,2270857395                                      & -2,5270857395             \\ \hline
Goiás               & 16.1                                              & 62.43                                                 & 93,33                                                     & 15,2500715166                                      & 0,8499284834              \\ \hline
Maranhão            & 20                                                & 50.32                                                 & 54,03                                                     & 21,9711379998                                      & -1,9711379998             \\ \hline
Mato Grosso         & 18.5                                              & 35.45                                                 & 85,67                                                     & 18,3882236801                                      & 0,1117763199              \\ \hline
Mato Grosso do Sul  & 13.9                                              & 39.5                                                  & 89,82                                                     & 17,4727152633                                      & -3,5727152633             \\ \hline
Minas Gerais        & 15.5                                              & 79.7                                                  & 88,69                                                     & 14,6550404545                                      & 0,8449595455              \\ \hline
Pará                & 20.6                                              & 51.51                                                 & 70,65                                                     & 19,4207165889                                      & 1,1792834111              \\ \hline
Paraíba             & 17.5                                              & 63.98                                                 & 84,01                                                     & 16,5156237909                                      & 0,9843762091              \\ \hline
Paraná              & 11.8                                              & 75.54                                                 & 92,39                                                     & 14,4158318771                                      & -2,6158318771             \\ \hline
Pernambuco          & 15.6                                              & 66.61                                                 & 84,74                                                     & 16,2121909247                                      & -0,6121909247             \\ \hline
Piauí               & 20.8                                              & 71.38                                                 & 61,83                                                     & 19,2518390307                                      & 1,5481609693              \\ \hline
Rio de Janeiro      & 14.1                                              & 90.46                                                 & 97,34                                                     & 12,574655617                                       & 1,525344383               \\ \hline
Rio Grande do Norte & 16.9                                              & 75.38                                                 & 86,43                                                     & 15,3106166025                                      & 1,5893833975              \\ \hline
Rio Grande do Sul   & 11.1                                              & 85.78                                                 & 92,31                                                     & 13,6673066389                                      & -2,5673066389             \\ \hline
Rondônia            & 17.1                                              & 72.76                                                 & 74,1                                                      & 17,3317103721                                      & -0,2317103721             \\ \hline
Roraima             & 15.4                                              & 88.55                                                 & 83,68                                                     & 14,7400518279                                      & 0,6599481721              \\ \hline
Santa Catarina      & 10.8                                              & 91.02                                                 & 92,23                                                     & 13,2900588893                                      & -2,4900588893             \\ \hline
São Paulo           & 11.6                                              & 95.82                                                 & 98,9                                                      & 11,9455505118                                      & -0,3455505118             \\ \hline
Sergipe             & 17.6                                              & 68.62                                                 & 84,39                                                     & 16,1147857545                                      & 1,4852142455              \\ \hline
Tocantins           & 19.3                                              & 37.13                                                 & 77,94                                                     & 19,4085829553                                      & -0,1085829553             \\ \hline
\end{tabular}%
}
\end{table}

\begin{table}[H]
\centering
\resizebox{\textwidth}{!}{%
\begin{tabular}{|l|c|c|c|c|c|c|}
\hline
\multicolumn{7}{|c|}{Resumo dos Resultados}                                                                                                                                          \\ \hline
Estatísticas da Regressão            &                       &                       &                       &                       &                       &                       \\ \hline
R Múltiplo                           & 0,7399894296          &                       &                       &                       &                       &                       \\ \hline
R Quadrado                           & 0,547584356           &                       &                       &                       &                       &                       \\ \hline
R Quadrado Ajustado                  & 0,5098830523          &                       &                       &                       &                       &                       \\ \hline
Erro Padrão                          & 2,4430871676          &                       &                       &                       &                       &                       \\ \hline
Observações                          & 27                    &                       &                       &                       &                       &                       \\ \hline
                                     &                       &                       &                       &                       &                       &                       \\ \hline
Análise de Variância                 &                       &                       &                       &                       &                       &                       \\ \hline
                                     & df                    & SS                    & MS                    & F                     & F de Significação     &                       \\ \hline
Regressão                            & 2                     & 173,3814318215        & 86,6907159107         & 14,5242817271         & 7,35277653639156E-005 &                       \\ \hline
Resíduo                              & 24                    & 143,2481978081        & 5,9686749087          &                       &                       &                       \\ \hline
Total                                & 26                    & 316,6296296296        &                       &                       &                       &                       \\ \hline
                                     & \multicolumn{1}{l|}{} & \multicolumn{1}{l|}{} & \multicolumn{1}{l|}{} & \multicolumn{1}{l|}{} & \multicolumn{1}{l|}{} & \multicolumn{1}{l|}{} \\ \hline
                                     & Coeficientes          & Erro Padrão           & Estatística T         & P-valor               & 95\% Inferiores       & 95\% Superiores       \\ \hline
Intercessão                          & 33,711583166          & 3,7884411322          & 8,8985368889          & 4,56093581088682E-009 & 25,8926249625         & 41,5305413695         \\ \hline
População coberta por esgotamento    & -0,0742554977         & 0,0305824409          & -2,4280435299         & 0,0230469361          & -0,1373745534         & -0,011136442          \\ \hline
População servida por coleta de lixo & -0,1481382292         & 0,0529031159          & -2,8001796591         & 0,0099250388          & -0,2573248941         & -0,0389515643         \\ \hline
\end{tabular}%
}
\end{table}

Sendo “t” o índice da observação e e[t] o erro padrão da regressão múltipla para cada observação, a expressão da regressão linear foi a seguinte:

\textbf{mort[t] = + 33.7116 -0.0742555esgot[t] -0.148138coleta[t] + e[t]}

Os coeficientes da regressão se mostram significativos, pois podemos observar que a estatística T de todos eles possuem valor absoluto superior a 2 e p-valor inferior a 0,05.

Nossa hipótese de que as variáveis de esgotamento e coleta de lixo explicam a mortalidade infantil se mantém válida, devido ao valor de F = 14,52428173 ser maior que o F Crítico
de 3,4 para 2 graus de liberdade do numerador e 24 graus de liberdade do denominador.

Interpretando os coeficientes de regressão em si, podemos retirar diversas informações pertinentes. Ambos os coeficientes das variáveis independentes (Porcentagens da população
coberta por esgotamento e por coleta de lixo) possuem uma relação inversa com a variável dependente (taxa de mortalidade infantil), ou seja, o aumento de valor de qualquer uma das
duas variáveis independentes deve levar à redução do valor da variável dependente.

Pela expressão da regressão linear também podemos inferir que o aumento de uma unidade na taxa de esgotamento leva à redução de 0,0742555 unidades na taxa de mortalidade e o aumento
de uma unidade na taxa de coleta de lixo leva à redução de 0,148138 unidades na taxa de mortalidade, além dos acréscimos devido ao erro.
